\documentclass{article}
\usepackage[margin=3cm]{geometry}

\usepackage{lmodern}
\usepackage[utf8]{inputenc} % allow utf-8 input
\usepackage[T1]{fontenc}    % use 8-bit T1 fonts
\usepackage{hyperref}       % hyperlinks
\usepackage{url}            % simple URL typesetting
\usepackage{booktabs}       % professional-quality tables
\usepackage{amsfonts}       % blackboard math symbols
\usepackage{nicefrac}       % compact symbols for 1/2, etc.
\usepackage{microtype}      % microtypography
\usepackage{xcolor}         % colors

\usepackage{bm}         % bold and italics maths font
\usepackage[utf8]{inputenc}
\usepackage{lipsum}
\usepackage{amssymb,amsthm,amsmath}
\usepackage{enumitem}
\usepackage{caption}
\usepackage{subcaption}
\usepackage{centernot}
\usepackage{complexity}
\usepackage[ruled,noline,linesnumbered,noend]{algorithm2e} % noend or vlined to remove end
\usepackage{mathrsfs}   % \mathscr fonts
\usepackage{tabularx}   % adjustable table width
\usepackage{multirow}   % multiple rows and columns
\usepackage{array}      % allows m{'width'} and b{'width'}
\usepackage{booktabs}   % \toprule \midrule \bottomrule
\usepackage{colortbl}   % allows table cells to be coloured
\usepackage{xargs}      % Use more than one optional parameter in a new commands
\usepackage{tikz}
\usepackage{tikz-cd}
\usepackage{marginnote} % https://tex.stackexchange.com/questions/297684/todonotes-on-both-sides-of-the-page
\usepackage{afterpage}
\usepackage{pdflscape}

\usetikzlibrary{positioning, calc}

% planning operators
\DeclareMathOperator*{\op}{op}
\DeclareMathOperator*{\pre}{pre}
\DeclareMathOperator*{\add}{add}
\DeclareMathOperator*{\del}{del}
\DeclareMathOperator*{\eff}{eff}

% complexity macros
\newclass{\N}{N}
\newclass{\CountingLogic}{C}
\newclass{\coNTIME}{coNTIME}
\newclass{\coNSPACE}{coNSPACE}
\newclass{\coNPSPACE}{coNPSPACE}
\newclass{\EXPTIME}{EXPTIME}
\newclass{\NEXPTIME}{NEXPTIME}
\newclass{\coNEXPTIME}{coNEXPTIME}
\newclass{\NEXPSPACE}{NEXPSPACE}
\newclass{\coNEXPSPACE}{coNEXPSPACE}
\newclass{\ASPACE}{ASPACE}
\newclass{\ATIME}{ATIME}
\newclass{\APSPACE}{APSPACE}
\newclass{\AEXPTIME}{AEXPTIME}
\newclass{\AEXPSPACE}{AEXPSPACE}

% maths headings
\newtheorem{assumption}{Assumption}
\newtheorem{theorem}{Theorem}[section]
\newtheorem{corollary}[theorem]{Corollary}
\newtheorem{lemma}[theorem]{Lemma}
\newtheorem{proposition}[theorem]{Proposition}
\theoremstyle{definition}
\newtheorem{definition}[theorem]{Definition}

% maths macros
\def\N{\mathbb{N}}
\def\R{\mathbb{R}}
\def\Z{\mathbb{Z}}

\def\a{\alpha}
\def\b{\beta}
\def\g{\gamma}
\def\d{\delta}
\def\e{\varepsilon}
\newcommand{\Alpha}{\mathrm{A}}
\newcommand{\Omicron}{\mathrm{O}}
\newcommand{\Tau}{\mathrm{T}}
\renewcommand{\phi}{\varphi}

\def\ra{\rightarrow}
\def\la{\leftarrow}
\DeclareMathOperator*{\argmax}{arg\,max}
\DeclareMathOperator*{\argmin}{arg\,min}

\newcommand{\abs}[1]{\left| #1 \right|}
\newcommand{\norm}[1]{\left\| #1 \right\|}
\newcommand{\gen}[1]{\left< #1 \right>}
\newcommand{\gena}[1]{\langle #1 \rangle}
\newcommand{\set}[1]{\left\{ #1 \right\}}
\newcommand{\seta}[1]{\{ #1 \}}
\newcommand{\setbig}[1]{\bigl\{ #1 \bigr\}}
\newcommand{\setbigg}[1]{\biggl\{ #1 \biggr\}}
\newcommand{\setBig}[1]{\Bigl\{ #1 \Bigr\}}
\newcommand{\mset}[1]{\left\{ \!\! \left\{ #1 \right\} \!\! \right\}}
\newcommand{\mseta}[1]{\{ \!\! \{ #1 \} \!\! \}}
\newcommand{\msetaa}[1]{\{ \!\! \{ \! #1 \! \} \!\! \}}
\newcommand{\msetbig}[1]{\bigl\{ \!\! \bigl\{ #1 \bigr\} \!\! \bigr\}}
\newcommand{\brk}[1]{\left[ #1 \right]}
\newcommand{\brka}[1]{\[ #1 \]}
\newcommand{\lr}[1]{\left( #1 \right)}
\newcommand{\lra}[1]{( #1 )}
\newcommand{\biglr}[1]{\bigl( #1 \bigr)}
\newcommand{\bigglr}[1]{\biggl( #1 \biggr)}
\newcommand{\Biglr}[1]{\Bigl( #1 \Bigr)}

\DeclareMathOperator*{\concat}{%
  \mathchoice%
    {\Big\Vert}%
    {\big\Vert}%
    {\Vert}%
    {\Vert}%
}

\DeclareMathOperator*{\concatsmall}{%
  {\Vert}%
}

% macros for tables
\newcommand{\header}[1]{\rotatebox[origin=l]{90}{\hspace*{-0.22cm} #1}}
\newcommand{\colorofcell}{gray}
\newcommand{\comparisonentry}[1]{{\tiny{#1}}}
\newcommand{\first}[2]{\cellcolor{\colorofcell!50}{{\textbf{#1}}\comparisonentry{#2}}}
\newcommand{\second}[2]{\cellcolor{\colorofcell!30}{{#1}\comparisonentry{#2}}}
\newcommand{\third}[2]{\cellcolor{\colorofcell!10}{{#1}\comparisonentry{#2}}}
\newcommand{\normalcell}[2]{{{#1}\comparisonentry{#2}}}
\newcommand{\zerocell}[1]{-}
\newcommand{\half}{0.495}

\newcolumntype{Y}{>{\raggedleft\arraybackslash}X}

% colours
\definecolor{caribbeangreen}{rgb}{0.0, 0.8, 0.6}
\definecolor{brilliantlavender}{rgb}{0.96, 0.73, 1.0}
\definecolor{amethyst}{rgb}{0.6, 0.4, 0.8}
\definecolor{ao(english)}{rgb}{0.0, 0.5, 0.0}
\definecolor{arylideyellow}{rgb}{0.91, 0.84, 0.42}
\definecolor{asparagus}{rgb}{0.53, 0.66, 0.42}
\definecolor{aquamarine}{rgb}{0.5, 1.0, 0.83}
\definecolor{babyblue}{rgb}{0.54, 0.81, 0.94}
\definecolor{fwtchanged}{rgb}{0.3, 0.3, 0.7}
\definecolor{rosewood}{rgb}{0.4, 0.0, 0.04}
\definecolor{oldmauve}{rgb}{0.4, 0.19, 0.28}
\definecolor{myrtle}{rgb}{0.13, 0.26, 0.12}
\definecolor{magenta(dye)}{rgb}{0.79, 0.08, 0.48}

\definecolor{plta}{rgb}{0.12, 0.47, 0.71}
\definecolor{pltb}{rgb}{   1, 0.5, 0.05}
\definecolor{pltc}{rgb}{0.17, 0.63, 0.17}
\definecolor{pltd}{rgb}{0.84, 0.15, 0.16}


\begin{document}
\begin{center}
    \Large
    LRNN for planning notes
\end{center}

\tableofcontents
\section{Related Work}
\begin{itemize}
    \item \url{https://doi.org/10.1016/S0004-3702(99)00060-0} [Khardon, AIJ 1999]
    \begin{itemize}
        \item early ILP work for learning rule based policies in Blocksworld and Logistics as domain control knowledge in the (old) GraphPlan planner
    \end{itemize}
    \item \url{https://arxiv.org/abs/2306.01439} [Delfosse et al., NIPS 2023]
    \begin{itemize}
        \item neural symbolic ILP for RL, the evaluation domains are simple game domains
    \end{itemize}
    \item \url{https://arxiv.org/abs/2106.11417} [Xu and Fekri, arxiv]
    \begin{itemize}
        \item learning action models for RL with ILP
    \end{itemize}
    \item \url{https://arxiv.org/abs/2304.08349} [Hazra and De Raedt, ECML PKDD 2023]
    \begin{itemize}
        \item neuro symbolic ILP for RL, tests for rewards in RL setting rather than satisfiability of plans
    \end{itemize}
\end{itemize}

\section{Problem Statements}
LRNNs and ILP can learn several useful things for planning.
\subsection{Learn Policies as Domain Knowledge}
\subsubsection*{Input} PDDL domain $D$.

\subsubsection*{Training Data} A list of the form $(P_1, T_1), \ldots, (P_n, T_n)$ where $P_i$ is a problem from domain $D$, and $T_i$ consists of tuples $(s, a, t)$ where $s$ is a state and $a$ is an applicable action in $s$ in $P_i$, and $t \in \set{0, 1}$ determines if $a$ is an optimal action for $s$.

\subsubsection*{Output} A policy $\pi$ that maps states to actions that hopefully generalises for $D$.

\subsection{Learn Transition Models}
\subsubsection*{Input} Predicate symbols and their arities.

\subsubsection*{Training Data} A set of tuples $(s, a, s')$ where $s$ is a state, $a$ is an action applicable in $s$, and $s'$ is the state that results from applying $a$ in $s$ from a set of planning problems. Furthermore, the set could be a complete or partial subset of all possible transitions in the set of planning problems.

\subsubsection*{Output}
A transition model $T$ that maps states to sets of states. A state is a set of ground atoms (instantiations of predicates by replacing predicate arguments with objects).

\section{Motivation}
\begin{enumerate}
    \item computational complexity argument: semi-positive datalog is P-complete
    \item if done correctly, policies are faster than heuristics as it bypasses evaluation on all successor states
    \item RL/robotics settings benefit from learning transition models
\end{enumerate}

\section{Domains}
\subsection{Blocksworld}
\begin{itemize}
    \item Blocks Clear: Blocksworld but only one clear goal
    \item Blocks On: Blocksworld but only one on goal
    \item General Blocksworld
\end{itemize}
\subsection{Transportation Domains}
\begin{itemize}
    \item Gripper: there are two rooms and a robot has to move all the \emph{balls} in one room to the other, and the robot can carry up to two balls at a time
    \item Ferry: there are $c$ \emph{cars} scattered across $l$ \emph{locations} and a ferry has to transport all the cars to their target location, and the ferry can only carry one car at a time
    \item Satellite: there is a set of \emph{satellites} that have \emph{instruments} that have certain \emph{modes} that can take pictures from a set of \emph{directions}; the goal is for the satellites to take a set of pictures of certain (mode, direction) configurations 
\end{itemize}
\subsection{Planning with Resources}
\begin{itemize}
    \item Spanner: a man is in a one-way corridor and has to fix some \emph{nuts} at the end of the corridor; he can pick up \emph{spanners} along the corridor but each spanner can only be used to fix one nut
    \item Childsnack: there are $c$ \emph{children}, some of which are allergic to gluten and the others not, which must be fed sandwiches that are gluten-free (GF) or not; the children allergic to gluten must be fed GF sandwiches while the others can eat any sandwich; there are a finite number of GF and non-GF \emph{ingredients} that can be used to make sandwiches
    \item Woodworking: see \url{https://github.com/potassco/pddl-instances/blob/master/ipc-2008/domains/woodworking-sequential-optimal-strips/domain.pddl} the idea is to treat some \emph{wood} to satisfy some target wood criteria
\end{itemize}
\subsection{Pathfinding}
\begin{itemize}
    \item Path finding on graph
\end{itemize}


\end{document}
